\documentclass[11pt]{article}
\usepackage{amsmath,amsfonts,amssymb,amsthm}

\usepackage{mathpazo}
\usepackage{fullpage}

\theoremstyle{plain}
\newtheorem*{claim}{Claim}


%%%%%%%%%%%%%%%%%%%%%%%%%%%%%%%%%%%

\theoremstyle{plain}
\newtheorem*{theorem}{Theorem}

\theoremstyle{remark}
\newtheorem*{solution}{Solution}

\def\naturals{\mathbb{N}}
\def\integers{\mathbb{Z}}
\def\rationals{\mathbb{Q}}
\def\reals{\mathbb{R}}
\def\complex{\mathbb{C}}


\usepackage[most,breakable]{tcolorbox}
\renewenvironment{boxed}%
	{\begin{tcolorbox}[colback=white,colframe=gray!10,breakable,enhanced]}%
	{\end{tcolorbox}}

\begin{document}


\title{MATH 135 --- Fall 2021\\ Sample Proofs from Lecture 10}
\author{Mark Girard}

\maketitle

\section*{Principle of Mathematical Induction}
Let $P(1),P(2),P(3),\cdots$ be a sequence of statements. (I.e., suppose $P(n)$ is an open sentence for numbers $n\in\naturals$.) If the following are true:
\begin{itemize}
\item[(i)] $P(1)$
\item[(ii)] $\forall k\in\naturals,\, P(k)\implies P(k+1)$
\end{itemize}
then it is also true that
\begin{itemize}
 \item[(iii)]$\forall n\in\naturals,\, P(n)$.
\end{itemize}




\subsubsection*{Example}

\begin{tcolorbox}
\begin{claim}
$ \forall n\in\naturals,\, 4\mid (5^n-1)$.
\end{claim}
\end{tcolorbox}

\begin{proof}
 We prove by induction. For each $n\in\naturals$,  let  $P(n)$ be the statement that $4\mid (5^n-1)$. 
 \begin{itemize}
  \item\underline{Base case:} When $n=1$, we have
  \[
   5^n-1 = 5- 1 = 4,
  \]
which is divisible by 4, so $P(1)$ is true.
\item\underline{\smash{Induction step}:} Let $k$ be an arbitrary natural number and suppose that $P(k)$ is true. That is, assume that there exists an integer $m$ such that 
\[
 4m = 5^k-1. \tag{IH}
\]
Now, 
\begin{align*}
 5^{k+1}-1 &= 5(5^k)-1 \\
           &= 5(4m+1) - 1 && \text{(by IH)}\\
           &= 4(5m+1),
\end{align*}
which is divisible by 4 as $5m+1$ is an integer, and thus $P(k+1)$ is true.
 \end{itemize}
By the principle of mathematical induction, it holds that $4\mid (5^n-1)$ for every $n\in\naturals$.
\end{proof}

\subsubsection*{Example}
\begin{tcolorbox}
\begin{claim}
 For every positive integer $n$, it holds that 
 \[
  \sum_{i=1}^n i(i+1) = \frac{n(n+1)(n+2)}{3}.
 \]

\end{claim}
\end{tcolorbox}
 
 
 
\begin{proof}
 We prove by induction. For each $n\in\naturals$,  let  $P(n)$ be the statement that \[\displaystyle\sum_{i=1}^ni(i+1)=\frac{n(n+1)(n+2)}{3}.\] 
 \begin{itemize}
  \item\underline{Base case:} When $n=1$, we have
  \[
   \sum_{i=1}^ni(i+1) = 1(1+1) = 2 = \frac{1\cdot 2\cdot 3}{3},
  \]
and thus $P(1)$ is true.
\item\underline{\smash{Induction step}:} Let $k$ be an arbitrary natural number and suppose that $P(k)$ is true. That is, assume that 
\[
\sum_{i=1}^ki(i+1)=\frac{k(k+1)(k+2)}{3}. \tag{IH}
\]
Now, 
\begin{align*}
 \sum_{i=1}^{k+1}i(i+1) 
            &= \sum_{i=1}^ki(i+1) + (k+1)(k+2) \\
           &= \frac{k(k+1)(k+2)}{3} + (k+1)(k+2)&& \text{(by IH)}\\
           &= (k+1)(k+2)\left(\frac{k}{3} + 1\right)\\
           &=\frac{(k+1)(k+2)(k+3)}{3}
\end{align*}
and thus $P(k+1)$ is true.
 \end{itemize}
By the principle of mathematical induction, it holds that $\displaystyle\sum_{i=1}^ni(i+1)=\frac{n(n+1)(n+2)}{3}$ for every $n\in\naturals$.
\end{proof}

\subsubsection*{Base case does not have to be $n=1$}


\begin{tcolorbox}
\begin{claim}
 For all $n\in\naturals$, if $n\geq 4$ then $n!>2^n$.
\end{claim}
\end{tcolorbox}
\begin{proof}
 We prove by induction. For each $n\in\naturals$,  let  $P(n)$ be the statement that $n!>2^n$. 
 \begin{itemize}
  \item\underline{Base case:} When $n=4$, we have
  \[
   4! = 1\cdot2\cdot3\cdot4 = 24 > 16 = 2^4,
  \]
and thus $P(4)$ is true.
\item\underline{\smash{Induction step}:} Let $k$ be an arbitrary natural number such that $k\geq4$. Suppose that $P(k)$ is true. That is, assume that 
\[
k!>2^k. \tag{IH}
\]
Now, 
\begin{align*}
 (k+1)! &= k!(k+1) \\
 & > 2^k(k+1)&&\text{(by IH)}\\
 & \geq 2^k\cdot 5 &&\text{(because }k\geq4)\\
 & > 2^k\cdot 2 &&\text{(because $5> 2$)}\\
 & = 2^{k+1}
\end{align*}
and thus $(k+1)!>2^{k+1}$, so $P(k+1)$ is true.
 \end{itemize}
By the principle of mathematical induction, it holds that $n!>2^n$ for every $n\geq4$ .
\end{proof}


\end{document}
