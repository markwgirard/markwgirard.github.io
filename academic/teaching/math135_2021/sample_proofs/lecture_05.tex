\documentclass[11pt]{article}
\usepackage{amsmath,amsfonts,amssymb,amsthm}

\usepackage{mathpazo}
\usepackage{fullpage}

\theoremstyle{plain}
\newtheorem*{claim}{Claim}


%%%%%%%%%%%%%%%%%%%%%%%%%%%%%%%%%%%

\theoremstyle{plain}
\newtheorem*{theorem}{Theorem}

\theoremstyle{remark}
\newtheorem*{solution}{Solution}

\def\naturals{\mathbb{N}}
\def\integers{\mathbb{Z}}
\def\rationals{\mathbb{Q}}
\def\reals{\mathbb{R}}
\def\complex{\mathbb{C}}


\usepackage[most,breakable]{tcolorbox}
\renewenvironment{boxed}%
	{\begin{tcolorbox}[colback=white,colframe=gray!10,breakable,enhanced]}%
	{\end{tcolorbox}}

\begin{document}


\title{MATH 135 --- Fall 2021\\ Sample Proofs from Lecture 5}
\author{Mark Girard}

\maketitle

\section*{Proving equalities}

To prove a universally quantified equality (``$\forall x\in S,\, f(x)= g(x)$''):

\begin{itemize}
 \item Start by letting the variables be arbitrary elements of the domain.
 \item Show that the LHS (left-hand side) is equal to the RHS (right-hand side) by writing out the expression of the LHS and manipulating it until you get the RHS.
 \item Make sure to explain or justify each each step that is not just a straightforward manipulation!
\end{itemize}

\begin{tcolorbox}
\begin{claim}
 For every $\theta\in\reals$, it holds that
 \[
  \sin(3\theta) = 3\sin\theta - 4\sin^3\theta.
 \]
\end{claim}
\end{tcolorbox}

\begin{proof}
 First recall the following trigonometric identities. For every choice of real numbers $\alpha,\beta\in\reals$, one has the following angle addition formula:
 \begin{equation*}\tag{$\ast$}\label{eq:angle_addition}
  \sin(\alpha+\beta) = \sin \alpha \cos \beta + \cos \alpha \sin \beta.
 \end{equation*}
For every real number $\alpha\in\reals$, one also has that
 \begin{align*}
\sin^2\alpha = \frac{1-\cos(2\alpha)}{2} \tag{$\ast\ast$}\label{eq:sin^2_identity}
 \end{align*}
 and that
 \begin{equation*}\label{eq:pythagorean_identity}
  \sin^2\alpha + \cos^2\alpha = 1. \tag{$\ast\ast\ast$}
 \end{equation*}
 Now let $\theta$ be an arbitrary real number. One has
 \begin{align*}
  \sin(3\theta) & = \sin(2\theta+\theta) \\
    & = \sin(2\theta)\cos\theta + \cos(2\theta)\sin\theta &&\text{(by the angle addition formula in \eqref{eq:angle_addition})}\\
    & = 2\sin\theta\cos^2\theta + \cos(2\theta)\sin\theta &&\text{(again by \eqref{eq:angle_addition})}\\
    & = 2\sin\theta\cos^2\theta + (1-2\sin^2\theta)\sin\theta && \text{(by \eqref{eq:sin^2_identity})}\\
    & = 2\sin\theta\cos^2\theta + \sin\theta-2\sin^3\theta &&\text{(by rearranging)}\\
    & = 2\sin\theta(1-\sin^2\theta) + \sin\theta-2\sin^3\theta &&\text{(by \eqref{eq:pythagorean_identity})}\\
    & = 3\sin\theta - 4\sin^3\theta,
 \end{align*}
as desired. 
\end{proof}

\section*{Proving inequalities}

Proving a universally quantified inequality (``$\forall x\in S,\, f(x)\geq g(x)$'' or ``$\forall x\in S,\, f(x)>g(x)$''):
\begin{itemize}
 \item Same idea as for equalities, but at each step the expression needs to be either equal to or greater than the next expression.
\end{itemize}

\begin{tcolorbox}
\begin{claim}
 For every $x\in\reals$ it holds that 
 \[
  x^2+5x+7 >0.
 \]
\end{claim}
\end{tcolorbox}
\begin{proof}
 Let $x$ be a real number. Now
 \begin{align*}
  x^2+5x+7 & = \left(x^2+5+\frac{25}{4}\right) - \frac{25}{4} + 7\\
           & = \left(x+\frac{5}{2}\right)^2 + \frac{3}{4}\\
           & \geq \frac{3}{4} && \text{(because all squares of real numbers are non-negative)}\\
           & >0,
 \end{align*}
which completes the proof.
\end{proof}

\section*{Proof by cases}
You can sometimes prove a statement by:
\begin{enumerate}
 \item Dividing the situation into cases which exhaust all the possibilities; and
 \item Showing that the statement follows in all cases.
\end{enumerate}


\begin{tcolorbox}
\begin{claim}
 For every choice of real numbers $x,y\in\reals$, it holds that
 \[
  \max\{x,y\} = \frac{x+y+|x-y|}{2}.
 \]
\end{claim}
\end{tcolorbox}
\begin{proof}
Let $x$ and $y$ be real numbers.  There are two cases to consider: either $x\geq y$ or $x<y$.
\begin{itemize}
\setlength{\itemindent}{2em}
 \item[Case 1:] Suppose $x\geq y$ such that $x-y\geq0$ and thus $|x-y| = x-y$. One has that
 \begin{align*}
  \max\{x,y\} = x = \frac{2x}{2} = \frac{x+y + x - y}{2} = \frac{x+y + |x-y|}{2},
 \end{align*}
as desired.
\item[Case 1:] Suppose $x< y$ such that $x-y<0$ and thus $|x-y| = -x+y$. One has that
 \begin{align*}
  \max\{x,y\} = y = \frac{2y}{2} = \frac{x+y - x + y}{2} = \frac{x+y + |x-y|}{2},
 \end{align*}
as desired.
\end{itemize}
This proves the claim, as the equality has been shown to hold in every possible case.
\end{proof}



\end{document}
