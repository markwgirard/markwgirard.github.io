\documentclass[11pt]{article}
\usepackage{amsmath,amsfonts,amssymb,amsthm}

\usepackage{mathpazo}
\usepackage{fullpage}

\theoremstyle{plain}
\newtheorem*{claim}{Claim}


%%%%%%%%%%%%%%%%%%%%%%%%%%%%%%%%%%%

\theoremstyle{plain}
\newtheorem*{theorem}{Theorem}

\theoremstyle{remark}
\newtheorem*{solution}{Solution}

\def\naturals{\mathbb{N}}
\def\integers{\mathbb{Z}}
\def\rationals{\mathbb{Q}}
\def\reals{\mathbb{R}}
\def\complex{\mathbb{C}}


\usepackage[most,breakable]{tcolorbox}
\renewenvironment{boxed}%
	{\begin{tcolorbox}[colback=white,colframe=gray!10,breakable,enhanced]}%
	{\end{tcolorbox}}

\begin{document}


\title{MATH 135 --- Fall 2021\\ Sample Proofs from Lecture 8}
\author{Mark Girard}

\maketitle

\section*{Proof by contrapositive}

A universally quantified statement of the form
\[
 \forall x\in\mathcal{S},\, P(x)\implies Q(x)
\]
is logically equivalent to its contrapositive:
\[
 \forall x\in\mathcal{S},\, \neg Q(x)\implies \neg P(x).
\]
For some universally quantified statements, it is easier to prove the contrapositive than to prove the original statement directly. 

\begin{tcolorbox}
\begin{claim}
$ \forall x\in\reals,\, x^3 - 5x^2 + 3x \neq 15 \implies x\neq 5.$
\end{claim}
\end{tcolorbox}

\begin{proof}
 We prove the contraspositive. Let $x$ be a real numbers and suppose that $x=5$. Then
 \[
  x^3 - 5x^2 + 3x = 5^3 - 5\cdot 5^2 + 3\cdot 5 = 5^3 - 5^3 - 15 = 15,
 \]
which completes the proof.
\end{proof}

\begin{tcolorbox}
\begin{claim}
 For all integers $k$, if $k^2+4k-2$ is odd then $k$ is odd.
\end{claim}
\end{tcolorbox}
\begin{proof}
 We prove the contrapositive. Let $k$ be an integer and suppose that $k$ is even. There exists an integer $m$ such that $k=2m$. Now,
 \[
  k^2+4k-2 = (2m)^2 + 4k - 2 = 2(2m^2 + 2k - 1),
 \]
which is even as $2m^2 + 2k - 1$ is an integer.
\end{proof}

\begin{tcolorbox}
\begin{claim}
 For all real numbers $x$ and $y$, if $xy$ is irrational then $x$ is irrational or $y$ is irrational. 
\end{claim}
\end{tcolorbox}
Symbollically, this claim can be written as:
\[
 \forall x,y\in\reals,\, xy\notin \rationals \implies (x\notin\rationals\vee y\notin\rationals).
\]
The contrapositive of this statement is 
\[
 \forall x,y\in\reals,\,  (x\in\rationals\wedge y\in\rationals \implies xy\in \rationals).
\]
\begin{proof}
 We prove the contrapositive. Let $x$ and $y$ be real numbers ans suppose that both $x$ and $y$ are rational. There exist integers $a,b,m,$ and $n$ such that $b\neq0$ and $n\neq0$ and 
 \[
  x = \frac{a}{b}\quad \text{and} \quad y = \frac{m}{n}.
 \]
Now
\[
 xy = \frac{a}{b}\cdot \frac{m}{n} = \frac{am}{bn},
\]
where $am$ and $bn$ are integers and $bn\neq 0$ as both $b$ and $n$ are nonzero. We conclude that $xy$ is rational, which completes the proof.
\end{proof}



\section*{Proof by Method of Elimination}

For sentences $A$, $B$, and $C$, it can be shown that
\[
 \bigl(A\implies (B\vee C)\bigr)\equiv \bigl((A\wedge \neg B)\implies C\bigr).
\]
That is, to prove that either $B$ or $C$ is true, we can suppose $B$ is false, which `eliminates' the possibility of $B$ being true, and then prove in this case that $C$ must be true.

For universally quantified statements, this looks like:
\[
 \Bigl(\forall x\in S,\bigl(P(x)\implies (Q(x)\vee R(x))\bigr)\Bigr)\equiv \Bigl(\forall x\in S,\bigl((P(x)\wedge \neg Q(x))\implies R(x)\bigr)\Bigr).
\]



\begin{tcolorbox}
\begin{claim}
 For all real numbers $x$, if $|2x-6|=4$ then $x\geq3$ or $x=1$.
\end{claim}
\end{tcolorbox}


\begin{proof}
We prove this statement by the Method of Elimination. Let $x$ be a real number and suppose that $|2x-6|=4$. Suppose further that $x<3$. Then $2x\leq 6$ and thus $2x-6\leq 0$ which implies that $|2x-6|=6-2x$. It follows that from the assumption that $|2x-6|=4$ that 
\[
 6-2x=4
\]
and solving this equation for $x$ yields $x=1$, as desired.
\end{proof}



\section*{Proving ``if and only if'' statements}

To prove a statement of the form $A\iff B$, one must prove both $A\implies B$ and $B\implies A$. For universally quantified statements, this equivalence is 
\[
 \Bigl(\forall x\in S,\, P(x)\iff Q(x)\Bigr)\equiv \Bigl(\forall x\in S,\, (P(x)\implies Q(x)) \wedge (Q(x)\implies P(x))\Bigr).
\]


\begin{tcolorbox}
\begin{claim}
 Suppose $x$ and $y$ are real numbers such that $x\geq0$ and $y\geq 0$. Then $\frac{x+y}{2} = \sqrt{xy}$ if and only if $x=y$.
\end{claim}
\end{tcolorbox}
\begin{proof}
 First suppose that $x=y$. Then 
 \[
  \frac{x+y}{2} = \frac{x+x}{2} = x = \sqrt{x\cdot x} = \sqrt{xy}.
 \]
Conversely, suppose instead that $\frac{x+y}{2} = \sqrt{xy}$. Multiplying both sides by 2 yields 
\[
 x+y = 2\sqrt{xy}.
\]
Squaring both sides, we find that 
\[
 x^2+ 2xy + y^2 = 4xy,
\]
which is equivalent to 
\[
 x^2-2xy+y^2 =0
\]
and thus 
\[
 (x-y)^2 = 0.
\]
We conclude that $x-y=0$ and thus $x=y$. This completes the proof.
\end{proof}

Note that $B\implies A$ is equivalent to $\neg A\implies \neg B$. Hence sometimes it is easier to prove 
\[
 A\implies B \quad \text{and}\quad \neg A\implies \neg B,
\]
which also proves equivalence.

\begin{tcolorbox}
\begin{claim}
 For all integers $a$, one has that $a$ is even if and only if $a^2$ is even.
\end{claim}
\end{tcolorbox}
\begin{proof}
 Let $a$ be an integer. First suppose that $a$ is even, such that there is an integer $k$ satisfying $2k=a$. Thus 
 \[
  a^2 = 4k^2 = 2\cdot(2k^2),
 \]
which is even as $2k^2$ is an integer. Conversely, suppose that $a$ is odd such that there is an integer $k$ satisfying $2k+1=a$. Thus
\[
  a^2 = (2k+1)^2 = 4k^2 + 4k + 1 = 2\cdot(2k^2 + 2k) + 1,
 \]
 which is odd as $2k^2 + 2k$ is an integer. This completes the proof.
\end{proof}

\end{document}
