\documentclass[11pt]{article}
\usepackage{amsmath,amsfonts,amssymb,amsthm}

\usepackage{mathpazo}
\usepackage{fullpage}

\theoremstyle{plain}
\newtheorem*{claim}{Claim}


%%%%%%%%%%%%%%%%%%%%%%%%%%%%%%%%%%%

\theoremstyle{plain}
\newtheorem*{theorem}{Theorem}

\theoremstyle{remark}
\newtheorem*{solution}{Solution}

\def\naturals{\mathbb{N}}
\def\integers{\mathbb{Z}}
\def\rationals{\mathbb{Q}}
\def\reals{\mathbb{R}}
\def\complex{\mathbb{C}}


\usepackage[most,breakable]{tcolorbox}
\renewenvironment{boxed}%
	{\begin{tcolorbox}[colback=white,colframe=gray!10,breakable,enhanced]}%
	{\end{tcolorbox}}

\begin{document}


\title{MATH 135 --- Fall 2021\\ Sample Proofs from Lecture 6}
\author{Mark Girard}

\maketitle

\section*{Proving existentially quantified statements}

To prove a existentially statement (``$\forall x\in S,\, P(x)$''):

\begin{itemize}
 \item Do the scratchwork investigation to determine a value of $x$ that works.
 \item Start your proof with ``Let $x=\dots$'' and state clearly that $x$ is in fact an element of $S$ (or demonstrate that $x$ is an element of $S$ if it is not clear).
 \item Then proceed to show that $P(x)$ is true for this particular value of $x$.
 \item Do not include your scratchwork as part of your proof! However, your proof must be self-contained, so include any explanations necessary to justify your argument.

\end{itemize}

\begin{tcolorbox}
\begin{claim}
 $\exists m\in\integers,\, \frac{m-7}{2m+4}=5$.
\end{claim}
\end{tcolorbox}

\begin{proof}
 Let $m=-3$, which is an integer. Now
 \begin{equation*}
  \frac{m-7}{2m+4} = \frac{-3-7}{-6+4} = \frac{-10}{-2} = 5,
 \end{equation*}
 which proves the claim.

\end{proof}

\begin{tcolorbox}
\begin{claim}
 There exists a perfect square $k$ such that $k^2-\frac{31}{2}k = 8$.
\end{claim}
\end{tcolorbox}
\begin{proof}
 Let $k=16$, which is a perfect square as $16=4^2$. Then
 \begin{align*}
  k^2 - \frac{31}{2}k &= 16^2 - 31\cdot 8\\& = 256 - 248 \\&= 8, 
 \end{align*}
as desired.
\end{proof}


\section*{Disproving statements}

To prove that a statement is false:
\begin{itemize}
 \item Negate the statement.
 \item Prove that the negation is true.
\end{itemize}



\begin{tcolorbox}
\begin{claim}
 $\forall x\in\reals,\, (x^2-1)^2>0$.
\end{claim}
\end{tcolorbox}
We prove this statement is false by stating its negation and proving that. The negation of this statement is:
\[
 \exists x\in\reals,\, (x^2-1)^2\leq0.
\]

\begin{proof}[Proof (of the negation)]
 Let $x=1$, which is a real number. Then
 \[
  (x^2-1)^2 = (1-1)^2 = 0^2 = 0 \leq 0,
  \]
as desired.
\end{proof}

\begin{tcolorbox}
\begin{claim}
 There exists a real number $\theta$ for which it holds that $\sin(2\theta)+\cos(2\theta)=3$.
\end{claim}
\end{tcolorbox}
We prove this statement is false by stating its negation and proving that. The negation of this statement is:
\[
 \forall\theta\in\reals,\, \sin(2\theta)+\cos(2\theta)\neq3.
\]
\begin{proof}[Proof (of the negation)]
 Let $\theta$ be a real number. Note that
 \begin{align*}
 &-1\leq \sin(2\theta)\leq 1\\ \text{and }&-1\leq \cos(2\theta)\leq 1,
 \end{align*}
and thus 
\[
 -2 \leq \sin(2\theta)+\cos(2\theta) \leq 2 <3.
\]
Hence $\sin(2\theta)+\cos(2\theta) < 3$.
\end{proof}

\section*{Proving statements with nested quantifiers}


\begin{tcolorbox}
Prove or disprove the following statements:
\begin{align*}
 A\colon &\text{``}\forall x\in\reals, \exists y\in\reals, x^3-y^3 =1\text{''} \\
 B\colon &\text{``}\exists y\in\reals, \forall x\in\reals, x^3-y^3 =1\text{''}
\end{align*}

\end{tcolorbox}

Statement $A$ is true but statement $B$ is false. In statement $A$, one decides on a value for $y$ \emph{after} a value for $x$ is given. In statement $B$, a value of $y$ is picked first and this value must work for every possible choice of $x$.

We first prove statement $A$.
\begin{proof}
 Let $x$ be a real number. Choose $y=(x^3-1)^{1/3}$, which is a real number. Then
 \[
  x^3-y^3 = x^3-\left((x^3-1)^{1/3}\right)^3 = x^3 - (x^3-1) = 1,
 \]
which completes the proof.
\end{proof}

Now state the negation of $B$:
\[
 \neg B \colon \text{``}\forall y\in\reals, \exists x\in\reals, x^3-y^3 \neq1\text{''}
\]
\begin{proof}[Proof (of negation of $B$)]
 Let $y$ be a real number and choose $x = (y^3-2)^{1/3}$. Then
 \[
  x^3-y^3 = \left((y^3-2)^{1/3}\right)^3- y^3 = y^3 - 2 + y^3 = 2 ,
 \]
which is not equal to 1.
\end{proof}


\section*{Proving implications}
An implication is a statement of the form ``$A\implies B$'' or ``$\forall x\in S,\, P(x)\implies Q(x)$.''
\begin{enumerate}
 \item Assume that the hypothesis (i.e., $A$ or $P(x)$) is true.
 \item Prove the conclusion (i.e., $B$ or $Q(x)$) using only what you know to be true.
 \item Do not worry about instances where the hypothesis is false!
\end{enumerate}


\begin{tcolorbox}
\begin{claim}
 For every integer $k$, if $k^5$ is a perfect square then $9k^{19}$ is a perfect square.
\end{claim}
\end{tcolorbox}
\begin{proof}
Let $k$ be an integer. Assume that $k^5$ is a perfect square. There exists an integer $m$ such that $m^2=k$. Now,
\begin{align*}
 9k^{19} &= 3^2\cdot k^{14} \cdot k^5 \\&= 3^2\cdot (k^7)^2\cdot m^2 \\&= (3mk^7)^2
\end{align*}
which is a perfect square as $3mk^7$ is an integer.
\end{proof}

\begin{tcolorbox}
\begin{claim}
 For every integer $n$, if $2^{2n}$ is odd then $2^{-2n}$ is odd.
\end{claim}
\end{tcolorbox}
\begin{proof}
 Let $n$ be an integer. There are three possible cases to consider: $n<0$, $n=0$, and $n>0$.
 \begin{itemize}
\setlength{\itemindent}{2em}
 \item[Case 1:] Suppose $n<0$. Then $2^{2n}$ is not an integer and thus not odd.
 \item[Case 2:] Suppose $n=0$. Then $2^{2n}=2^0 = 1$, which is odd. In this case, one has  $2^{-2n} = 2^0 =1$, which is again odd.
\item[Case 2:] Suppose $n>0$. Then $n-1\geq0$ and thus 
\[2^{2n} = 2^{2n-2+2} = 2^2\cdot 2^{2(n-1)}  = 2(2\cdot 4^{n-1})\]
which is even as $2\cdot 4^{n-1}$ is an integer, and thus $2^{2n}$ is not odd.
 \end{itemize}
 This proves the claim, as the implication has been shown to be true in every case where the hypothesis holds.
\end{proof}



\end{document}
