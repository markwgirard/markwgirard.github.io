\documentclass[11pt]{article}
\usepackage{amsmath,amsfonts,amssymb,amsthm}

\usepackage{mathpazo}
\usepackage{fullpage}

\theoremstyle{plain}
\newtheorem*{claim}{Claim}


%%%%%%%%%%%%%%%%%%%%%%%%%%%%%%%%%%%

\theoremstyle{plain}
\newtheorem*{theorem}{Theorem}

\theoremstyle{remark}
\newtheorem*{solution}{Solution}

\def\naturals{\mathbb{N}}
\def\integers{\mathbb{Z}}
\def\rationals{\mathbb{Q}}
\def\reals{\mathbb{R}}
\def\complex{\mathbb{C}}


\usepackage[most,breakable]{tcolorbox}
\renewenvironment{boxed}%
	{\begin{tcolorbox}[colback=white,colframe=gray!10,breakable,enhanced]}%
	{\end{tcolorbox}}

\begin{document}


\title{MATH 135 --- Fall 2021\\ Sample Proofs from Lecture 11}
\author{Mark Girard}

\maketitle

\section*{Principle of Strong Induction}
When assuming only the one previous step is not enough to prove the next step, Strong Induction is needed! 
The principle of strong induction can be stated more generally as follows.
\begin{tcolorbox}
\textbf{Principle of Strong Induction.}
Let $b\in\naturals$ and suppose $P(b),\, P(b+1),\, P(b+2),\, \dots$ is a sequence of statements. Let $c\in\naturals$. If the following are true:
\begin{itemize}
\item[(i)] $P(b),P(b+1),\dots,P(b+c-1)$, and
\item[(ii)] For all $k\geq b+c-1$, if $P(b)\wedge P(b+1) \wedge\cdots\wedge P(k)$ then $P(k+1)$
\end{itemize}
then it is also true that
\begin{itemize}
 \item[(iii)]For all $n\geq b$,  $P(n)$. 
\end{itemize}
\end{tcolorbox}
Note that the base cases are $P(b),P(b+1),\dots,P(b+c-1)$, where
\begin{itemize}
 \item $b$ is the \emph{first} base case (there must always be at least one base case), and 
 \item $c$ is the \emph{total number of base cases}.
\end{itemize}



\subsubsection*{Example: Recursively defined sequences}
(Note: The example I did in class is on the homework, so I've typed up a different example here.)
\begin{tcolorbox}
Consider a sequence $x_1,x_2,x_3,\dots$ defined as follows. Define $x_1=2$ and $x_2 = 34$, and for every integer $n\geq 2$ define
\[
 x_{n+1} = 2x_n + 15x_{n-1}.
\]
It holds that 
\[
 x_n = (-3)^n  + 5^{n} \tag{$\ast$}
\]
for every $n\in\naturals$.
\end{tcolorbox}

\begin{proof}
 We prove by induction. For each $n\in\naturals$,  let  $P(n)$ be the statement that $x_n = (-3)^n  + 5^{n}$. 
 \begin{itemize}
  \item\underline{Base cases:} When $n=1$, we have
  \[
   x_1 = 2 = -3 + 5 = (-3)^1 + 5^1,
  \]
  so $P(1)$ is true. When $n=2$, we have
  \[
   x_2 = 34 = 9 + 25 = (-3)^2 + 5^2,
  \]
  so $P(2)$ is true.
\item\underline{\smash{Induction step}:} Let $k\in\naturals$ and suppose $k\geq 2$.  Assume for every $m\in\{1,2,\dots,k\}$ that $P(m)$ is true. That is, we assume that for every $m\in\{1,2,\dots,k\}$ it holds that
\[
 x_m = (-3)^m  + 5^{m}. \tag{IH}
\]
Now, because $k\geq2$ we may use $(\ast)$ to write
\begin{align*}
 x_{k+1} &= 2x_k + 15x_{k-1} &&\text{by }(\ast)\\ 
  & = 2\left((-3)^k  + 5^{k}\right)  + 15\left((-3)^{k-1}  + 5^{k-1}\right) &&\text{by IH}\\
  & = 2\cdot (-3)^k + 2\cdot 5^k - 5\cdot (-3)^k + 3\cdot 5^k\\
  & = (-3)\cdot(-3)^k + 5\cdot 5^k\\\
  & = (-3)^{k+1}  + 5^{k+1},
\end{align*}
 and thus $P(k+1)$ is true.
 \end{itemize}
By the principle of strong induction, it follows that $x_n = (-3)^n  + 5^{n}$ for every $n\in\naturals$.
\end{proof}

\subsubsection*{Example: Breaking up a chocolate bar}
Note that using strong induction does not necessarily mean that we need multiple base cases. Here is an example of a proof that requires strong induction but has only one base case.

\begin{tcolorbox}
\begin{claim}
For every $n\in\naturals$, breaking a rectangular chocolate bar consisting of $n$ identical squares requires $n-1$ breaks to completely break into individual pieces.
\end{claim}
\end{tcolorbox}
 
 
 
\begin{proof}
 We prove by induction. For each $n\in\naturals$,  let  $P(n)$ be the statement: 
 \begin{quotation}
  \noindent ``Every rectangular chocolate bar consisting of $n$ identical squares requires $n-1$ breaks to completely break into individual pieces.''
 \end{quotation}

 \begin{itemize}
  \item\underline{Base case:} When $n=1$, every rectangular chocolate bar of $1$ square is already completely broken apart and requires zero breaks, so $P(1)$ is true.
\item\underline{\smash{Induction step}:} Let $k$ be an arbitrary natural number and suppose that $P(1),P(2),\dots$ and $P(k)$ are all true. Now, suppose you are given a rectangular chocolate bar consisting of $k+1$ identical squares. As $k\geq1$, the chocolate bar requires at least one break to break apart. Suppose we break the bar into two smaller rectangles such that the two smaller rectangles consist of  $a$ and $b$ squares respectively, where $a,b\in\{1,2,\dots,k\}$. By the induction hypothesis, breaking each of these two smaller rectangles requires a total of $a-1$ and $b-1$ breaks respectively. Thus, the total number of breaks to break the original chocolate bar of size $k+1$ squares into individual pieces is 
\begin{align*}
 \underbrace{1}_{\text{for the first break}} + \underbrace{(a-1)}_{\text{to break the bar of size }a} + \underbrace{(b-1)}_{\text{to break the bar of size }b} &= \underbrace{a+b}_{=k+1} - 1 \\
  &= (k+1)-1,
\end{align*}
and thus $P(k+1)$ is true.
 \end{itemize}
By the principle of mathematical induction, it holds that $P(n)$ is true for every $n\in\naturals$. This completes the proof.
\end{proof}


\end{document}
