\documentclass[11pt]{article}
\usepackage{amsmath,amsfonts,amssymb,amsthm}

\usepackage{mathpazo}
\usepackage{fullpage}

\theoremstyle{plain}
\newtheorem*{claim}{Claim}


%%%%%%%%%%%%%%%%%%%%%%%%%%%%%%%%%%%

\theoremstyle{plain}
\newtheorem*{theorem}{Theorem}

\theoremstyle{remark}
\newtheorem*{solution}{Solution}

\def\naturals{\mathbb{N}}
\def\integers{\mathbb{Z}}
\def\rationals{\mathbb{Q}}
\def\reals{\mathbb{R}}
\def\complex{\mathbb{C}}


\usepackage[most,breakable]{tcolorbox}
\renewenvironment{boxed}%
	{\begin{tcolorbox}[colback=white,colframe=gray!10,breakable,enhanced]}%
	{\end{tcolorbox}}

\begin{document}


\title{MATH 135 --- Fall 2021\\ Sample Proofs from Lecture 9}
\author{Mark Girard}

\maketitle

\section*{Proof by Contradiction}
\begin{itemize}
 \item Given a statement $A$, \emph{\textbf{exactly one}} of $A$ and $\neg A$ is true. 
 \item A \emph{contradiction} is a statement of the form:
\[
A\wedge \neg A.
\tag{$\ast$}\label{eq:contradiction}
\]
A statement of the form in \eqref{eq:contradiction} must be false!
\item If you make an assumption in a proof, and using logical reasoning you are able to use that assumption to arrive at a contradiction of the form $A\wedge \neg A$, then your original assumption must have been wrong!
\end{itemize}

 \noindent To prove a statement $P$ by contradiction:
 \begin{enumerate}
 \item Suppose instead that $P$ is false (i.e., that $\neg P$ is true).
 \item Use the assumption that $\neg P$ is true to arrive at a contradiction of the form $A\wedge \neg A$.
 \item Conclude that the assumption that $P$ is false must have been wrong.
 \item This proves that $P$ is true.
\end{enumerate}



\subsubsection*{Example}

\begin{tcolorbox}
\begin{claim}
$ \forall a,b\in\integers,\, a\geq 2\implies (a\nmid b \text{ or }a\nmid (b+1))$.
\end{claim}
\end{tcolorbox}
In words, this says that ``No integer greater than one can divide two successive integers.'' 
\begin{proof}
 Let $a$ and $b$ be integers and assume that $a\geq 2$. [We will prove that either $a\nmid b$ or $a\mid (b+1)$.] For the sake of deriving a contradiction, suppose instead that $a\mid b$ and $a\nmid (b+1)$. Then there exist integers $m$ and $n$ such that 
 \[
  b=am\quad\text{and}\quad b+1 = an.
 \]
Then $am=b = an-1$ and thus $a(n-m) = 1$, which implies that $a\mid 1$, as $n-m$ is an integer. Because the only integers that divide 1 are 1 and $-1$, this implies that either $a=1$ or $a=-1$ and thus $a<2$ in either case. We conclude that both
\[
 a\geq2\quad\text{and}\quad a<2,
\]
which is a contradiction. Thus the assumption that $a\mid b$ and $a\mid (b+1)$ is false. Hence it must be the case that either $a\nmid b$ or $a\nmid (b+1)$. This completes the proof.
\end{proof}

Note that the proof of the above claim is essentially equivalent to proving by contrapositive. Either method is fine here.

\subsubsection*{Proof of irrationality of $\sqrt{2}$}
Here is an example of a proof by contradiction that \emph{cannot} be redone as a proof by contrapositive.

\begin{tcolorbox}
\begin{claim}
 $\sqrt 2$ is irrational
\end{claim}
\end{tcolorbox}
 
 
 
\begin{proof}
 Towards a contradiction, suppose instead that $\sqrt{2}$ were rational. Then there exist integers $a$ and $b$ having no common divisors (other than $1$ and $-1$) such that $b\neq0$ and 
 \begin{equation}
  \sqrt{2} = \frac{a}{b}. \label{eq:sqrt2ab}
 \end{equation}
 As we may suppose that $a$ and $b$ are reduced and have no common factors, we may conclude that they are not both even. (Otherwise, we would have that $2\mid a$ and $2\mid b$, which would mean that $a$ and $b$ share 2 as a common factor.) Squaring both sides of \eqref{eq:sqrt2ab} and rearranging, we find that 
\[
 2b^2 = a^2
\]
and thus $a^2$ is even, which implies that $a$ is even. Hence there is an integer $k$ such that $a=2k$. Now
\[
 2b^2 = (2k)^2 = 4k^2
\]
and thus $b^2=2k^2$, which implies that $b^2$ is also even and thus $b$ is even. We conclude that both $a$ and $b$ are even, which contradicts the statement that $a$ and $b$ are chosen such that they have no common factors. Hence, the assumption that $\sqrt{2}$ is rational is false. It follows that $\sqrt{2}$ must be irrational.
\end{proof}

\subsubsection*{Another example}


\begin{tcolorbox}
\begin{claim}
 For all real numbers $x$, if $x>0$ then $x+\frac{1}{x}\geq2$.
\end{claim}
\end{tcolorbox}
\begin{proof}
 Let $x$ be a real number ans suppose that $x>0$. Suppose for the sake of obtaining a contradiction that 
 \[
  x+\frac{1}{x}<2.
 \]
Multiplying both sides by $x$ and rearranging yields
\[
 x^2 - 2x + 1 < 0
\]
or equivalently $(x-1)^2<0$. But the square of every real number is non-negative, so it is also the case that  $(x-1)^2\geq0$. Hence we conclude that both
\[
 (x-1)^2<0 \qquad\text{and}\qquad(x-1)^2\geq0
\]
are true, which is a contradiction. Therefore our assumption that $x+\frac{1}{x}<2$ is false, which proves that $x+\frac{1}{x}\geq2$, as desired.
\end{proof}


\subsection*{Proving uniqueness}

To prove a statement of the form:
\[
 \text{``There exists a unique }x\in S\text{ such that }P(x)\text{ is true''}
\]
we must prove two things:
\begin{enumerate}
 \item[(i)]  Prove there exists at least one $x\in S$ such that $P(x)$ is true.
 \item[(ii)] Prove that, if $y\in S$ is another element such that $P(y)$ is true, then it must be that $y=x$.
\end{enumerate}
Symbolically, these two statements are:
\begin{enumerate}
 \item[(i)]  $\exists x\in S,\, P(x)$.
 \item[(ii)] $\forall y\in S,\, P(y)\implies (y=x)$.
\end{enumerate}


\subsubsection*{Example}
\begin{tcolorbox}
\begin{claim}
 For every odd integer $a$, there exists a unique integer $k$ such that $a=2k+1$.
\end{claim}
\end{tcolorbox}
\begin{proof}
 Let $a$ be an odd integer. By definition, there exists an integer $k$ such that $a=2k+1$. Suppose now that $m$ is another integer such that $a=2m+1$. Then
 \[
  2k+1 = 2m+1
 \]
which implies that $k=m$. Thus, $k$ is the unique integer satisfying this claim.
\end{proof}


\end{document}
